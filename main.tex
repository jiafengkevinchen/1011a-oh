\documentclass[11pt]{article}
\usepackage{macros}
\singlespacing
\begin{document}

\section{Externality and Bargaining}

Kevin is a graduate student in economics and is endowed with a dollar amount
$y_A$ per day.  Francisco is not an economist and is endowed with dollar amount
$y_B$ per day. It costs Kevin \$4 an hour to produce economics research. Let $x$
denote the number of hours per day that Kevin devotes to research. Kevin's
research gives Francisco trading signals, and thus Francisco also has a
preference over Kevin's research. Kevin's and Francisco's utilities are $u_A,
u_B$, respectively:

\begin{align*}
  u_A(x, y_A) &= \begin{cases}
  y_A + 8x - x^2/2 & x \le 8 \\
  y_A + 32 & o.w.
\end{cases} \\
  u_B(x, y_B) &= \begin{cases}
  y_B + 12x - x^2/2 & x \le 12 \\
  y_B + 72 & o.w.
\end{cases}
\end{align*}
    
\begin{enumerate}

% Problem 1
\item What are the marginal rates of substitution between Francisco and Kevin?

% Problem 2
\item How much will Kevin produce if he doesn't know that Francisco exists.

% Problem 3
\item Suppose that Kevin produces the amount of research in 2 and Francisco has
found a way to consume Kevin's research by downloading it for free from ArXiv.
Kevin has no way to prevent Francisco's free-riding. How much consumer surplus
do Francisco and Kevin obtain?

% Problem 4
\item Pareto efficient allocations are characterized by the relation, ``sum of
marginal rates of substitution equals marginal cost of production.'' What is the
Pareto efficient volume of research for Kevin to produce? What is the total
surplus at this level of research?

% Problem 5
\item Now suppose that Kevin has found a way to protect his research from
Francisco's free-loading. From now on, Francisco will have to pay $p$ dollars
for every unit of research he downloads. If Kevin chooses the price $p$ that
maximizes his ``profit from Francisco'' (revenue from Francisco, minus cost of
production), what price will he charge and how much research will he produce?
What will be Kevin and Francisco's consumer surpluses?


\end{enumerate}

\section{Cobb--Douglas}

\section{Vertical Monopolies}
There is a unit mass of identical price-taking consumers. Each consumer has income $y$ and consumes widgets $x$ and some num\'eraire good $x_0$, with utility function \[
u(x, x_0) = x_0 + \frac{1}{\sigma+1} x^{\sigma+1}, \quad -1 < \sigma < 0
\]
The price of $x$ is $p$ and the price of $x_0$ is $1$.
\begin{enumerate}
    \item Assume that $y$ is large so that $x_0, x$ have positive demand. Derive the Marshallian demand for $x$ as a function of $p$. Given $p$, derive the consumer surplus in good $x$.
    \begin{sol}
        The first order conditions are \[
        x^{\sigma} = \lambda p \quad \lambda = 1.
        \]
        Thus $x = p^{1/\sigma}$ and $x_0 = y - p^{\frac{\sigma + 1}{\sigma}}$, assuming $y > p^{\frac{\sigma + 1}{\sigma}}$. 
        
        The consumer surplus is \[
        \text{CS} = \int_{p}^{\infty} t^{1/\sigma}\,dt = \frac{1}{1+1/\sigma}t^{1/\sigma + 1}\Bigg \rvert_{p}^\infty = \frac{-\sigma}{\sigma + 1} p^{1 + 1/\sigma}.
        \]
    \end{sol}
    
     \begin{sol}\textbf{[Alternative Solution]}
          Recognize that the Marshallian demand for a quasilinear utility function $u(a,b) = a + v(b)$ with $p_a=1, p_b=p$ is \[b = (v')^{-1}(p).\]
              \end{sol} 
    \item What's the elasticity of demand?
    \begin{sol}
        \[
        \diff{\log x}{ \log p} = \frac{1}{\sigma}. 
        \]
    \end{sol}
    \item 
    Assume the consumers purchase positive amounts of $x$.
    The consumers purchase $x$ from a monopolist retailer, who sets the price $p$. Suppose the retailer faces constant marginal cost $c_R$, solve for the profit-maximizing $p$ as a function of $c_R$. Derive the retailer's profit $\pi_R$. 
    \begin{sol}
         The monopolist maximizes \[
         \max_p \, p^{1+1/\sigma} - c_R p^{1/\sigma} 
         \]
         Thus \[
         \frac{1+\sigma}{\sigma} p^{1/\sigma} - \frac{1}{\sigma}c_R p^{\frac{1-\sigma}{\sigma}} = 0 \implies p = \frac{1}{1+\sigma} c_R. 
         \]
         
         The retailer's profit is \[
         \pi_R = \pr{\frac{1}{1+\sigma} c_R }^{1 + 1/\sigma} - c_R\pr{\frac{1}{1+\sigma} c_R }^{1/\sigma}
         \]
         
     \end{sol}
     \begin{sol}\textbf{[Alternative Solution]}
          Recognize that the monopolist sets price with the markup rule \[
          \text{Price} = \pr{\frac{1}{1+1/\epsilon}} \cdot \pr{\text{Marginal Cost}} = \frac{1}{1+\sigma} c_R,
          \]
          where $\epsilon < 0$ is the elasticity of demand.
      \end{sol} 
     \item Assume the monopolist purchases $x$ from a monopolistic wholesaler, who sets $c_R$ and faces constant, exogenous marginal cost $c_W$. Solve for the profit-maximizing $c_W$. Derive the wholesaler's profit $\pi_W$ and the sum of the wholesaler and retailer's profits $\pi_W + \pi_R$.
     \begin{sol}
     Note that \[
     x = \pr{\frac{1}{1+\sigma} c_R }^{1/\sigma}
     \]
         The wholesaler solves \[
         \max_p \, c_R \pr{\frac{1}{1+\sigma}c_R }^{1/\sigma} - c_W \pr{\frac{1}{1+\sigma} c_R}^{1/\sigma}
         \]
    By the problem above, we have that \[c_R = \frac{1}{1+\sigma} c_W.\] 
    
    Let $A = \frac{1}{1+\sigma}$. We have \begin{align*}
        \pi_W &= A^{1/\sigma} \pr{c_R^{1+1/\sigma} - c_W c_R^{1/\sigma}}
        \\
        &=  A^{1/\sigma} \pr{A^{1+1/\sigma} c_W^{1+1/\sigma} - A^{1/\sigma}c_W c_W^{1/\sigma}} \\
        &= A^{2/\sigma} \pr{A-1}c_W^{1+1/\sigma}\\
        \pi_W + \pi_R &= \pr{A^2 c_W}^{1 + 1/\sigma} - A c_W \pr{A^2 c_W}^{1/\sigma} + A^{2/\sigma} \pr{A-1}c_W^{1+1/\sigma} \\
        &= A^{2+2/\sigma} c_W^{1+1/\sigma} - A^{1+2/\sigma} c_W^{1+1/\sigma} + A^{1 + 2/\sigma} c_W^{1+1/\sigma} - A^{2/\sigma}c_W^{1+1/\sigma} \\
        &= A^{2/\sigma} (A+1)(A-1) c_W^{1+1/\sigma}.
    \end{align*}
    
     \end{sol}
     
     \item What is the total surplus $\text{CS} + \pi_W + \pi_R$?
     \begin{sol}
     \begin{align*}
     \text{Total Surplus} &= A^{2/\sigma} (A+1)(A-1) c_W^{1+1/\sigma} - \frac{\sigma}{\sigma + 1} \pr{A^2 c_W}^{1 + 1/\sigma} \\
     &= c_W^{1+1/\sigma} \pr{A^{2/\sigma} (A+1)(A-1) - \sigma A^{3+2/\sigma}}
     \end{align*}
     \end{sol}
     
     \item Suppose the retailer and wholesaler merge into a monopoly that charges $p_{\text{merged}}$ to the consumers and takes $c_W$ marginal cost. Find the optimal $p_{\text{merged}}$ and total surplus.  
     \begin{sol}
         $p_{\text{merged}} = Ac_W.$ The monopoly's profit is \[
         \pr{Ac_W}^{1+1/\sigma} - c_W (Ac_W)^{1/\sigma} = A^{1/\sigma}(A-1) c_W^{1+1/\sigma}. 
         \]
         \begin{align*}
         \text{Total Surplus} &=    A^{1/\sigma}(A-1) c_W^{1+1/\sigma} - \frac{\sigma}{\sigma+1} \pr{Ac_W}^{1+1/\sigma} \\
         &= c_W^{1+1/\sigma} \pr{A^{1/\sigma}(A-1) -\sigma A^{2+1/\sigma}}
         \end{align*}
         
     \end{sol}
     \item Suppose $\sigma = -1/2$. Should the total-surplus-maximizing Federal Trade Commission let the retailer and wholesaler merge?
     
    \begin{sol}
             Consider \begin{align*}
             \frac{\text{TS}_\text{separate}}{\text{TS}_{\text{merged}}} &= \frac{ c_W^{1+1/\sigma} \pr{A^{2/\sigma} (A+1)(A-1) - \sigma A^{3+2/\sigma}}}{c_W^{1+1/\sigma} \pr{A^{1/\sigma}(A-1) -\sigma A^{2+1/\sigma}}}\\
             &= A^{1/\sigma} \pr{A + \frac{A-1}{-\sigma A^2 + A -1}} \\
             \end{align*}
            Plugging in $\sigma = -1/2$, $A = 2$, we have that the ratio equals $\frac{1}{4}\pr{2 + \frac{1}{2 + 2 - 1}} = \frac{7}{12} < 1$. Therefore the FTC should let the firms merge. 
         \end{sol}     
    
    \item (Bonus) Prove that the total surplus under the merged firm is larger for all $\sigma \in (-1,0)$. 
    [Hint: Use a version of Bernoulli's inequality: $(1+t)^r \ge rt$ for $t > -1$, $r \le 0$] 
    \begin{sol}
    Observe that \[
    -\sigma A^2 + A - 1 = \frac{-\sigma + (1+\sigma) - (1+\sigma)^2}{(1+\sigma)^2} = \frac{- 2 \sigma - \sigma^2 }{(1+\sigma)^2} = \frac{-\sigma(\sigma+2)}{(1+\sigma)^2}
    \]
        Consider \begin{align*}
             \frac{\text{TS}_\text{separate}}{\text{TS}_{\text{merged}}} &= \frac{ c_W^{1+1/\sigma} \pr{A^{2/\sigma} (A+1)(A-1) - \sigma A^{3+2/\sigma}}}{c_W^{1+1/\sigma} \pr{A^{1/\sigma}(A-1) -\sigma A^{2+1/\sigma}}}\\
             &= A^{1/\sigma} \pr{A + \frac{A-1}{-\sigma A^2 + A -1}} \\
              \\
             &= A^{1/\sigma} \pr{A + \frac{(1+\sigma)^2}{-\sigma (\sigma + 2)} \frac{-\sigma}{1+\sigma}} \\
             &= A^{1+1/\sigma} \pr{1 + \frac{(1+\sigma)^2}{\sigma+2}} \\
             &=  A^{1+1/\sigma} \pr{\frac{\sigma^2 + 3 \sigma + 3}{\sigma + 2}}.
             \end{align*}
             By the inequality $(1+x)^r \ge 1 + rx$, we have 
             \[
             A^{1+1/\sigma} = \frac{1}{(1+\sigma)^{1+1/\sigma}} \le \frac{1}{2+\sigma}.
             \]
             Thus \[
             \frac{\text{TS}_\text{separate}}{\text{TS}_{\text{merged}}} \le \frac{\sigma^2 + 3 \sigma + 3}{(\sigma+2)^2} = \frac{\sigma^2 + 3 \sigma + 3}{\sigma^2 + 3 \sigma + (4 + \sigma)} < 1
             \]
             for any $\sigma \in (-1,0)$. 
    \end{sol}
    \item True/False/Uncertain: Mergers always hurt the consumer.
\end{enumerate}
\end{document}
